\documentclass[10pt, letterpaper]{article}

% Packages:
\usepackage[
    ignoreheadfoot, % set margins without considering header and footer
    top=1.2 cm, % seperation between body and page edge from the top
    bottom=2 cm, % seperation between body and page edge from the bottom
    left=1.8 cm, % seperation between body and page edge from the left
    right=1.8 cm, % seperation between body and page edge from the right
    footskip=1.0 cm, % seperation between body and footer
    % showframe % for debugging 
]{geometry} % for adjusting page geometry
\usepackage[explicit]{titlesec} % for customizing section titles
\usepackage{tabularx} % for making tables with fixed width columns
\usepackage{array} % tabularx requires this
\usepackage[dvipsnames]{xcolor} % for coloring text
\definecolor{primaryColor}{RGB}{0, 79, 144} % define primary color
\usepackage{enumitem} % for customizing lists
\usepackage{fontawesome5} % for using icons
\usepackage{amsmath} % for math
\usepackage[
    pdftitle={Federico Bello's CV},
    pdfauthor={Federico Bello},
    pdfcreator={LaTeX with RenderCV},
    colorlinks=true,
    urlcolor=primaryColor
]{hyperref} % for links, metadata and bookmarks
\usepackage[pscoord]{eso-pic} % for floating text on the page
\usepackage{calc} % for calculating lengths
\usepackage{bookmark} % for bookmarks
\usepackage{lastpage} % for getting the total number of pages
\usepackage{changepage} % for one column entries (adjustwidth environment)
\usepackage{paracol} % for two and three column entries
\usepackage{ifthen} % for conditional statements
\usepackage{needspace} % for avoiding page brake right after the section title
\usepackage{iftex} % check if engine is pdflatex, xetex or luatex

% Ensure that generate pdf is machine readable/ATS parsable:
\ifPDFTeX
    \input{glyphtounicode}
    \pdfgentounicode=1
    \usepackage[T1]{fontenc}
    \usepackage[utf8]{inputenc}
    \usepackage{lmodern}
\fi

\usepackage[default, type1]{sourcesanspro} 

% Some settings:
\AtBeginEnvironment{adjustwidth}{\partopsep0pt} % remove space before adjustwidth environment
\pagestyle{empty} % no header or footer
\setcounter{secnumdepth}{0} % no section numbering
\setlength{\parindent}{0pt} % no indentation
\setlength{\topskip}{0pt} % no top skip
\setlength{\columnsep}{0.15cm} % set column seperation
\makeatletter
\let\ps@customFooterStyle\ps@plain % Copy the plain style to customFooterStyle
\makeatother
\pagestyle{customFooterStyle}

\titleformat{\section}{
    % avoid page braking right after the section title
    \needspace{4\baselineskip}
    % make the font size of the section title large and color it with the primary color
    \Large\color{primaryColor}
}{
}{
}{
    % print bold title, give 0.15 cm space and draw a line of 0.8 pt thickness
    % from the end of the title to the end of the body
    \textbf{#1}\hspace{0.15cm}\titlerule[0.8pt]\hspace{-0.1cm}
}[] % section title formatting

\titlespacing{\section}{
    % left space:
    -1pt
}{
    % top space:
    0.3 cm
}{
    % bottom space:
    0.2 cm
} % section title spacing

% \renewcommand\labelitemi{$\vcenter{\hbox{\small$\bullet$}}$} % custom bullet points
\newenvironment{highlights}{
    \begin{itemize}[
        topsep=0.10 cm,
        parsep=0.10 cm,
        partopsep=0pt,
        itemsep=0pt,
        leftmargin=0.4 cm + 10pt
    ]
}{
    \end{itemize}
} % new environment for highlights

\newenvironment{highlightsforbulletentries}{
    \begin{itemize}[
        topsep=0.10 cm,
        parsep=0.10 cm,
        partopsep=0pt,
        itemsep=0pt,
        leftmargin=10pt
    ]
}{
    \end{itemize}
} % new environment for highlights for bullet entries


\newenvironment{onecolentry}{
    \begin{adjustwidth}{
        0.2 cm + 0.00001 cm
    }{
        0.2 cm + 0.00001 cm
    }
}{
    \end{adjustwidth}
} % new environment for one column entries

\newenvironment{twocolentry}[2][]{
    \onecolentry
    \def\secondColumn{#2}
    \setcolumnwidth{\fill, 4 cm}
    \begin{paracol}{2}
}{
    \switchcolumn \raggedleft \secondColumn
    \end{paracol}
    \endonecolentry
} % new environment for two column entries

\newenvironment{threecolentry}[3][]{
    \onecolentry
    \def\thirdColumn{#3}
    \setcolumnwidth{1 cm, \fill, 4.5 cm}
    \begin{paracol}{3}
    {\raggedright #2} \switchcolumn
}{
    \switchcolumn \raggedleft \thirdColumn
    \end{paracol}
    \endonecolentry
} % new environment for three column entries

\newenvironment{header}{
    \setlength{\topsep}{0pt}\par\kern\topsep\centering\color{primaryColor}\linespread{1.15}
}{
    \par\kern\topsep
} % new environment for the header

% save the original href command in a new command:
\let\hrefWithoutArrow\href

% new command for external links:
\renewcommand{\href}[2]{\hrefWithoutArrow{#1}{\ifthenelse{\equal{#2}{}}{ }{#2 }\raisebox{.15ex}{\footnotesize \faExternalLink*}}}


\begin{document}
    \newcommand{\AND}{\unskip
        \cleaders\copy\ANDbox\hskip\wd\ANDbox
        \ignorespaces
    }
    \newsavebox\ANDbox
    \sbox\ANDbox{}

    \begin{header}
        \fontsize{30 pt}{30 pt}
        \textbf{Federico Bello}

        \vspace{0.3 cm}

        \small
        \mbox{\hrefWithoutArrow{mailto:fe.debello13@gmail.com}{{\footnotesize\faEnvelope[regular]}\hspace*{0.13cm}fe.debello13@gmail.com}}%
        \kern 0.2 cm%
        \AND%
        \kern 0.2 cm%
        \mbox{\hrefWithoutArrow{tel:+59891658041}{{\footnotesize\faPhone*}\hspace*{0.13cm}+598 91 658 041}}%
        \kern 0.2 cm%
        \AND%
        \kern 0.2 cm%
        \mbox{\hrefWithoutArrow{https://fedebello.com/}{{\footnotesize\faLink}\hspace*{0.13cm}fedebello.com}}%
        \kern 0.2 cm%
        \AND%
        \kern 0.2 cm%
        \mbox{\hrefWithoutArrow{https://linkedin.com/in/federico-bello-}{{\footnotesize\faLinkedinIn}\hspace*{0.13cm}federico-bello-}}%
        \kern 0.2 cm%
        \AND%
        \kern 0.2 cm%
        \mbox{\hrefWithoutArrow{https://github.com/fede-bello}{{\footnotesize\faGithub}\hspace*{0.13cm}fede-bello}}%
    \end{header}



    \section{Summary}
\begin{onecolentry}
    I am a Telecommunications Systems Engineer (UdelaR) specialized in signal processing. I currently work as a Machine Learning Engineer at Tryolabs, where I focus on designing and deploying production ML systems. I also have experience as a university professor teaching programming and conducting research on DNA storage in the Information Theory Group.
\end{onecolentry}


\section{Experience}

    \begin{twocolentry}{
        Montevideo, Uruguay

    Oct 2024 -- Present
    }
        \textbf{Tryolabs}, Machine Learning Engineer
        \begin{highlights}
            \item Developed production-ready ML systems for international clients, covering end-to-end pipelines: data processing, model training, evaluation, and deployment.
            \item Contributed to internal initiatives including onboarding, technical interviews, and knowledge-sharing sessions.
        \end{highlights}
    \end{twocolentry}

    \vspace{0.1 cm}

    \begin{twocolentry}{
        Montevideo, Uruguay

    Feb 2023 -- Feb 2025
    }
        \textbf{UdelaR, School of Engineering (InCo)}, Teaching and Research Assistant
        \begin{highlights}
            \item Taught and coordinated programming courses with 100+ students, delivering lectures and designing labs for Programming I (imperative programming, algorithm design) and Programming II (data structures, modularization, algorithm analysis).  
            \item Collaborated in the Information Theory Group, contributing to research on DNA storage. Assisted with theoretical analysis, theorem proofs, and the publication.
        \end{highlights}
    \end{twocolentry}

    \vspace{0.1 cm}

    \begin{twocolentry}{
        Montevideo, Uruguay

    Feb 2024 -- Oct 2024
    }
        \textbf{Eagerworks}, Junior AI Developer
        \begin{highlights}
            \item Led backend development for Docshunter, a document management and generation platform, building a reusable API.
            \item Authored technical blog posts and participated in technical interviews.
        \end{highlights}
    \end{twocolentry}

    \vspace{0.1 cm}

    \begin{twocolentry}{
        Montevideo, Uruguay

    Aug 2023 -- Feb 2024
    }
        \textbf{Sinapsis}, Junior Developer
        \begin{highlights}
            \item Built web applications for risk analysis used by national and international banks.
        \end{highlights}
    \end{twocolentry}

        \section{Education}

        \begin{twocolentry}{
            Mar 2019 -- Jun 2025
        }
        \textbf{B.Sc., School of Engineering (UdelaR)}, Telecommunication Systems Engineering, Montevideo
        \begin{highlights}
            \item Specialization in Signal Processing and Machine Learning.
            \item Thesis: \textit{Time Series Anomaly Detection using Graph Neural Networks}, including development of the open-source framework \href{https://github.com/GraGODs/GraGOD}{GraGOD}.
        \end{highlights}
        \end{twocolentry}
        
        \vspace{0.1cm}
        
        \begin{twocolentry}{Jan 2018}
\textbf{Studio Cambridge}, Cambridge, England — Advanced English Course
\end{twocolentry}



    \section{Publications}
\begin{twocolentry}{
    Aug 2024
}
\textbf{The Lattice-Input Discrete-Time Poisson Channel} — IEEE ISIT, Athens, Greece\\
Federico Bello, \'{A}lvaro Mart\'\i n, Tatiana Rischewski, Gadiel Seroussi
\end{twocolentry}

\section{Extra-Curricular Activities}

\begin{twocolentry}{2021}
\textbf{GraduaDocentes} — Volunteer teacher
\end{twocolentry}

\vspace{0.1cm}

\begin{twocolentry}{2015--2018}
\textbf{Colegio La Mennais} — Student charity program
\end{twocolentry}




\section{Languages}

\begin{onecolentry}
\begin{highlightsforbulletentries}
    \item \textbf{Spanish} — Native
    \item \textbf{English} — C2 Proficiency
    \item \textbf{German} — B1 (Goethe Institute, 2022--Present)
\end{highlightsforbulletentries}
\end{onecolentry}

   

\end{document}
